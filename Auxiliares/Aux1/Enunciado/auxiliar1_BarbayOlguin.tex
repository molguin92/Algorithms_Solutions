\documentclass[11pt,spanish]{article}

\usepackage[utf8]{inputenc}
\usepackage[spanish]{babel}
\usepackage{amsfonts}
\usepackage{amsmath}

\usepackage{graphicx}

\pagestyle{empty}

\begin{document}
\title{\vspace*{-3cm}CC3001 Algoritmos y Estructuras de Datos
\\Auxiliar 1
}
\author{
Prof.\ Jérémy Barbay
\\Aux. \ Manuel Olguín
}

\date{Fecha: 9 de septiembre de 2014}

\maketitle
\thispagestyle{empty}

\section*{Problema 1: Iteratividad}

Para cada uno de los algoritmos requeridos, considere lo siguiente:

\begin{itemize}
  \item Defina un invariante adecuado.
  \item Defina las condiciones iniciales para que el invariante se cumpla antes de comenzar el primer ciclo.
  \item Defina la condición de término del algoritmo.
  \item Escriba el cuerpo del ciclo en Java.
\end{itemize}

Los algoritmos son:

\begin{enumerate}
 \item Dado un String, diseñe un algoritmo iterativo para determinar si éste es palíndromo. Un String es palíndromo si su inverso es igual a él (por ejemplo: ``sometemos'').
 \item Diseñe un algoritmo iterativo que invierta las consonantes en un String. Por ejemplo, si el algoritmo recibe como entrada el String ``paralelepípedo'', debe retornar ``dapapelelírepo''.
\end{enumerate}

\section*{Problema 2: Recursión}

La función de Ackermann es una función importante en la teoría de computabilidad. Se define de manera recursiva como sigue:

\[ A(m,n) =
\begin{cases}
    n+1& \text{si } m = 0\\
    A(m-1, 1)& \text{si } m > 0 \text{ y } n = 0\\
    A(m-1, A(m, n-1))& \text{si } m > 0 \text{ y } n > 0
\end{cases}
\]

Implementela en Java (obviamente, de manera recursiva). Además, implemente un método \texttt{main()} que permita obtener los parámetros $m$ y $n$ desde \texttt{STDIN} y luego retorne el resultado en \texttt{STDOUT}.

\end{document}
